\documentclass[11pt]{article}
\usepackage[margin=1in]{geometry}
\usepackage{graphicx}
\usepackage{caption}
\usepackage{subcaption}
\usepackage{hyperref}
\usepackage{booktabs}
\usepackage{array}
\usepackage{xcolor}
\usepackage{listings}
\usepackage{float}
\usepackage{parskip}

\hypersetup{
  colorlinks=true,
  linkcolor=blue,
  urlcolor=blue,
  citecolor=blue
}

% Graphics search paths
\graphicspath{{./}{./images/}}

% Code listing style
\lstset{
  basicstyle=\ttfamily\small,
  breaklines=true,
  frame=single,
  columns=fullflexible,
  showstringspaces=false,
  keywordstyle=\color{blue!60!black},
  commentstyle=\color{green!50!black},
  stringstyle=\color{red!70!black}
}

% Metadata
\newcommand{\StudentName}{Hanqi Jiang}
\newcommand{\StudentID}{811732819}
\newcommand{\Course}{Final Project Submission} % Updated context
\newcommand{\Term}{Fall 2025}
\newcommand{\ProjectTitle}{GameHub: The Intelligent Social Gaming Platform}

\begin{document}

% -----------------------------------------------------------------------------
% Header / Title Page
% -----------------------------------------------------------------------------
\begin{center}
    \vspace*{1cm}
    \Huge \textbf{\ProjectTitle} \\
    \vspace{0.5cm}
    \Large \textbf{Preliminary Project Report} \\
    \vspace{2cm}
    
    \large
    \textbf{Student Name:} \StudentName \\
    \textbf{Student ID:} \StudentID \\
    \textbf{Course:} \Course \\
    \textbf{Term:} \Term \\
    
    \vspace{2cm}
    \today
\end{center}

\newpage
\tableofcontents
\newpage

% -----------------------------------------------------------------------------
% 1. Problem and Domain
% -----------------------------------------------------------------------------
\section{Title and Project Identity}
\textbf{Project Name:} GameHub \\
\textbf{Tagline:} The Ultimate Intelligent Social Platform for Gamers.

\section{Problem and Domain Description}

\subsection{The Domain}
The project operates within the \textbf{Digital Entertainment and Social Online Networks} domain. The video game industry has evolved from solitary offline experiences to highly interconnected, community-driven ecosystems. Modern players do not just "consume" content; they curate collections, share detailed reviews, track achievements, and seek personalized recommendations from peers.

\subsection{The Problem}
Despite the abundance of games released annually (over 10,000 on Steam alone in 2023), players face two critical issues:
\begin{enumerate}
    \item \textbf{Discovery Paralysis}: Finding the "right" game is increasingly difficult. Traditional storefronts offer generic "top sellers" lists but often lack personalized, context-aware discovery mechanisms that understand a player's specific tastes beyond simple genre tags.
    \item \textbf{Fragmented Social Experience}: The social aspect of gaming (reviews, discussions, community validation) is often disconnected from the management experience (library, wishlist). Players frequently switch between Reddit for discussions, Steam/Epic for library management, and sites like Metacritic for reviews, creating a disjointed user journey.
\end{enumerate}

% -----------------------------------------------------------------------------
% 2. Solution and User Interfaces
% -----------------------------------------------------------------------------
\section{Proposed Solution}

\subsection{The Core Solution}
\textbf{GameHub} acts as a unified web application that serves as a single destination for discovery, tracking, and social engagement. We address the discovery problem using \textbf{Artificial Intelligence} (LLM-based Chatbots and Review Summarizers) and solve the fragmentation problem by integrating a rich social feed directly into the user's personal library.

\subsection{Key User Interfaces}
The solution features a responsive, modern Web Interface with the following core views:

\begin{itemize}
    \item \textbf{The Discovery Dashboard (Home)}: 
    A visual grid of games featuring advanced filtering (by Genre, Platform, Release Year) and a unique "Smart Recommendation" row. This view is designed to reduce decision fatigue.
    
    \item \textbf{The Game Detail Portal}: 
    A comprehensive view for a single game. It integrates:
        \begin{itemize}
            \item \textbf{AI Review Summary}: A dynamically generated "Pros & Cons" list aggregated from hundreds of community reviews using GPT-5.1.
            \item \textbf{Social Actions}: Buttons to Rate, Write Reviews, Like existing reviews, and manage Library status (Owned/Wishlist).
        \end{itemize}

    \item \textbf{Personal Library Interface}: 
    A highly customizable table view where users can track their gaming detailed progress. Fields include "Status" (Playing, Completed, Dropped), "Play Time", and "Completion Percentage". It serves as the user's digital trophy case.

    \item \textbf{The "Game Master" Chat Widget}: 
    A persistent, floating AI assistant located at the bottom-right of the screen. Users can expand it to ask for real-time advice (e.g., "Is this game hard?", "Show me similar RPGs") without leaving the current page.
\end{itemize}

% Placeholder for UI Screenshot
\begin{figure}[H]
    \centering
    \fbox{\parbox{0.9\textwidth}{\centering \vspace{3cm} \textbf{[PLACEHOLDER: Insert Screenshot of Home Dashboard]} \\ \small{Showing Game Grid, Search Bar, and User Profile Dropdown}}} \vspace{3cm}}}
    \caption{Proposed UI: The Discovery Dashboard}
    \label{fig:ui_dashboard}
\end{figure}

% -----------------------------------------------------------------------------
% 3. Preliminary ER Diagram
% -----------------------------------------------------------------------------
\section{Preliminary ER Diagram}

The database is designed using a relational model to handle complex many-to-many relationships between users, games, and their interactions.

\subsection{Entity Sets}
\begin{enumerate}
    \item \textbf{USER}: Represents the platform's registered members. Attributes include \texttt{user\_id}, \texttt{username}, \texttt{password\_hash}, \texttt{email}, and profile metadata.
    \item \textbf{GAME}: Represents the video games available for tracking. Attributes include \texttt{game\_id}, \texttt{title}, \texttt{genre}, \texttt{developer}, and \texttt{metacritic\_score}.
    \item \textbf{USER\_GAME (Library)}: A junction entity representing the state of a game in a user's library. Attributes: \texttt{status}, \texttt{playtime\_hours}, \texttt{completion\_percentage}.
    \item \textbf{REVIEW}: User-generated content. Attributes: \texttt{rating}, \texttt{review\_text}, \texttt{helpful\_count}.
    \item \textbf{ACTIVITY}: Represents social feed events (e.g., "User X liked Review Y").
    \item \textbf{PLATFORM} and \textbf{GENRE}: Reference entities for categorization.
\end{enumerate}

\subsection{Relationships}
\begin{itemize}
    \item \textbf{User Library Tracking (M:N)}: Resolved by the \texttt{USER\_LIBRARY} table. A single User tracks many Games; a single Game is tracked by many Users.
    \item \textbf{Social Engagement (1:M)}: A User generates many \texttt{ACTIVITY} records.
    \item \textbf{Review Feedback (1:M)}: A User writes many Reviews; a Game receives many Reviews.
    \item \textbf{Review Interaction (M:N)}: Resolved by \texttt{REVIEW\_LIKE}. A User can like many Reviews.
    \item \textbf{Categorization (M:N)}: Games belong to multiple \texttt{GENRE}s and \texttt{PLATFORM}s.
\end{itemize}

% Placeholder for ER Diagram
\begin{figure}[H]
    \centering
    \fbox{\parbox{0.8\textwidth}{\centering \vspace{4cm} \textbf{[PLACEHOLDER: Insert ER Diagram Image Here]} \\ \small{Showing relationships between USER, GAME, REVIEW, and LIBRARY}}} \vspace{4cm}}}
    \caption{Entity-Relationship Diagram illustrating the relational schema.}
    \label{fig:er_diagram}
\end{figure}

% -----------------------------------------------------------------------------
% 4. Technologies Used
% -----------------------------------------------------------------------------
\section{Technologies Used}

We utilize a modern, industry-standard stack to ensure scalability, performance, and maintainability.

\begin{table}[H]
\centering
\renewcommand{\arraystretch}{1.5}
\begin{tabular}{|p{0.25\textwidth}|p{0.7\textwidth}|}
\hline
\textbf{Layer} & \textbf{Technologies} \\
\hline
\textbf{Frontend (UI)} & \textbf{Next.js 14 (React)}: Utilized for server-side rendering and efficient routing. \newline \textbf{Tailwind CSS}: For rapid, responsive styling. \\
\hline
\textbf{Backend (API)} & \textbf{Spring Boot 3.2 (Java 17)}: Provides robust RESTful services. \newline \textbf{Spring Security}: Integrated with \textbf{JWT} for stateless authentication. \\
\hline
\textbf{Database} & \textbf{MySQL 8.0}: Relational data persistence, running in a Docker container for isolation. \\
\hline
\textbf{AI Services} & \textbf{Python (Flask)}: A lightweight microservice. \newline \textbf{OpenAI API}: Powers the GPT-4o integration for chat and summary generation. \\
\hline
\textbf{DevOps} & \textbf{Docker Compose}: Orchestrates the multi-container environment. \newline \textbf{Git}: Version control. \\
\hline
\end{tabular}
\caption{Project Technology Stack}
\end{table}

\end{document}
